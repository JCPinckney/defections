% Options for packages loaded elsewhere
% Options for packages loaded elsewhere
\PassOptionsToPackage{unicode}{hyperref}
\PassOptionsToPackage{hyphens}{url}
\PassOptionsToPackage{dvipsnames,svgnames,x11names}{xcolor}
%
\documentclass[
  letterpaper,
  DIV=11,
  numbers=noendperiod]{scrartcl}
\usepackage{xcolor}
\usepackage{amsmath,amssymb}
\setcounter{secnumdepth}{-\maxdimen} % remove section numbering
\usepackage{iftex}
\ifPDFTeX
  \usepackage[T1]{fontenc}
  \usepackage[utf8]{inputenc}
  \usepackage{textcomp} % provide euro and other symbols
\else % if luatex or xetex
  \usepackage{unicode-math} % this also loads fontspec
  \defaultfontfeatures{Scale=MatchLowercase}
  \defaultfontfeatures[\rmfamily]{Ligatures=TeX,Scale=1}
\fi
\usepackage{lmodern}
\ifPDFTeX\else
  % xetex/luatex font selection
  \setmainfont[]{Garamond}
  \setsansfont[]{Garamond}
\fi
% Use upquote if available, for straight quotes in verbatim environments
\IfFileExists{upquote.sty}{\usepackage{upquote}}{}
\IfFileExists{microtype.sty}{% use microtype if available
  \usepackage[]{microtype}
  \UseMicrotypeSet[protrusion]{basicmath} % disable protrusion for tt fonts
}{}
\makeatletter
\@ifundefined{KOMAClassName}{% if non-KOMA class
  \IfFileExists{parskip.sty}{%
    \usepackage{parskip}
  }{% else
    \setlength{\parindent}{0pt}
    \setlength{\parskip}{6pt plus 2pt minus 1pt}}
}{% if KOMA class
  \KOMAoptions{parskip=half}}
\makeatother
% Make \paragraph and \subparagraph free-standing
\makeatletter
\ifx\paragraph\undefined\else
  \let\oldparagraph\paragraph
  \renewcommand{\paragraph}{
    \@ifstar
      \xxxParagraphStar
      \xxxParagraphNoStar
  }
  \newcommand{\xxxParagraphStar}[1]{\oldparagraph*{#1}\mbox{}}
  \newcommand{\xxxParagraphNoStar}[1]{\oldparagraph{#1}\mbox{}}
\fi
\ifx\subparagraph\undefined\else
  \let\oldsubparagraph\subparagraph
  \renewcommand{\subparagraph}{
    \@ifstar
      \xxxSubParagraphStar
      \xxxSubParagraphNoStar
  }
  \newcommand{\xxxSubParagraphStar}[1]{\oldsubparagraph*{#1}\mbox{}}
  \newcommand{\xxxSubParagraphNoStar}[1]{\oldsubparagraph{#1}\mbox{}}
\fi
\makeatother


\usepackage{longtable,booktabs,array}
\usepackage{calc} % for calculating minipage widths
% Correct order of tables after \paragraph or \subparagraph
\usepackage{etoolbox}
\makeatletter
\patchcmd\longtable{\par}{\if@noskipsec\mbox{}\fi\par}{}{}
\makeatother
% Allow footnotes in longtable head/foot
\IfFileExists{footnotehyper.sty}{\usepackage{footnotehyper}}{\usepackage{footnote}}
\makesavenoteenv{longtable}
\usepackage{graphicx}
\makeatletter
\newsavebox\pandoc@box
\newcommand*\pandocbounded[1]{% scales image to fit in text height/width
  \sbox\pandoc@box{#1}%
  \Gscale@div\@tempa{\textheight}{\dimexpr\ht\pandoc@box+\dp\pandoc@box\relax}%
  \Gscale@div\@tempb{\linewidth}{\wd\pandoc@box}%
  \ifdim\@tempb\p@<\@tempa\p@\let\@tempa\@tempb\fi% select the smaller of both
  \ifdim\@tempa\p@<\p@\scalebox{\@tempa}{\usebox\pandoc@box}%
  \else\usebox{\pandoc@box}%
  \fi%
}
% Set default figure placement to htbp
\def\fps@figure{htbp}
\makeatother


% definitions for citeproc citations
\NewDocumentCommand\citeproctext{}{}
\NewDocumentCommand\citeproc{mm}{%
  \begingroup\def\citeproctext{#2}\cite{#1}\endgroup}
\makeatletter
 % allow citations to break across lines
 \let\@cite@ofmt\@firstofone
 % avoid brackets around text for \cite:
 \def\@biblabel#1{}
 \def\@cite#1#2{{#1\if@tempswa , #2\fi}}
\makeatother
\newlength{\cslhangindent}
\setlength{\cslhangindent}{1.5em}
\newlength{\csllabelwidth}
\setlength{\csllabelwidth}{3em}
\newenvironment{CSLReferences}[2] % #1 hanging-indent, #2 entry-spacing
 {\begin{list}{}{%
  \setlength{\itemindent}{0pt}
  \setlength{\leftmargin}{0pt}
  \setlength{\parsep}{0pt}
  % turn on hanging indent if param 1 is 1
  \ifodd #1
   \setlength{\leftmargin}{\cslhangindent}
   \setlength{\itemindent}{-1\cslhangindent}
  \fi
  % set entry spacing
  \setlength{\itemsep}{#2\baselineskip}}}
 {\end{list}}
\usepackage{calc}
\newcommand{\CSLBlock}[1]{\hfill\break\parbox[t]{\linewidth}{\strut\ignorespaces#1\strut}}
\newcommand{\CSLLeftMargin}[1]{\parbox[t]{\csllabelwidth}{\strut#1\strut}}
\newcommand{\CSLRightInline}[1]{\parbox[t]{\linewidth - \csllabelwidth}{\strut#1\strut}}
\newcommand{\CSLIndent}[1]{\hspace{\cslhangindent}#1}



\setlength{\emergencystretch}{3em} % prevent overfull lines

\providecommand{\tightlist}{%
  \setlength{\itemsep}{0pt}\setlength{\parskip}{0pt}}



 


\usepackage{booktabs}
\usepackage{longtable}
\usepackage{array}
\usepackage{multirow}
\usepackage{wrapfig}
\usepackage{float}
\usepackage{colortbl}
\usepackage{pdflscape}
\usepackage{tabu}
\usepackage{threeparttable}
\usepackage{threeparttablex}
\usepackage[normalem]{ulem}
\usepackage{makecell}
\usepackage{xcolor}
\KOMAoption{captions}{tableheading}
\makeatletter
\@ifpackageloaded{caption}{}{\usepackage{caption}}
\AtBeginDocument{%
\ifdefined\contentsname
  \renewcommand*\contentsname{Table of contents}
\else
  \newcommand\contentsname{Table of contents}
\fi
\ifdefined\listfigurename
  \renewcommand*\listfigurename{List of Figures}
\else
  \newcommand\listfigurename{List of Figures}
\fi
\ifdefined\listtablename
  \renewcommand*\listtablename{List of Tables}
\else
  \newcommand\listtablename{List of Tables}
\fi
\ifdefined\figurename
  \renewcommand*\figurename{Figure}
\else
  \newcommand\figurename{Figure}
\fi
\ifdefined\tablename
  \renewcommand*\tablename{Table}
\else
  \newcommand\tablename{Table}
\fi
}
\@ifpackageloaded{float}{}{\usepackage{float}}
\floatstyle{ruled}
\@ifundefined{c@chapter}{\newfloat{codelisting}{h}{lop}}{\newfloat{codelisting}{h}{lop}[chapter]}
\floatname{codelisting}{Listing}
\newcommand*\listoflistings{\listof{codelisting}{List of Listings}}
\makeatother
\makeatletter
\makeatother
\makeatletter
\@ifpackageloaded{caption}{}{\usepackage{caption}}
\@ifpackageloaded{subcaption}{}{\usepackage{subcaption}}
\makeatother
\usepackage{bookmark}
\IfFileExists{xurl.sty}{\usepackage{xurl}}{} % add URL line breaks if available
\urlstyle{same}
\hypersetup{
  pdftitle={Conceptual Framework for Defections Project},
  pdfauthor={Jonathan Pinckney},
  colorlinks=true,
  linkcolor={blue},
  filecolor={Maroon},
  citecolor={Blue},
  urlcolor={Blue},
  pdfcreator={LaTeX via pandoc}}


\title{Conceptual Framework for Defections Project}
\author{Jonathan Pinckney}
\date{2026-02-04}
\begin{document}
\maketitle


\section{Introduction}\label{introduction}

When do supporters of authoritarian and democratic backsliding regimes
cease to support their leaders? When do they \emph{defect}? And what
actions by opposition forces, civil resistance campaigns, or third party
actors can motivate such changes? This document presents a conceptual
framework for understanding this process. The framework will guide a
series of research projects that seek to generate and test hypotheses on
when, why, and how defection happens. These research projects in turn
will inform a series of public engagement and storytelling projects to
facilitate understanding of how to spark defection from authoritarian
and democratic backsliding regimes.

\subsection{What is Defection?}\label{what-is-defection}

All political regimes rely on key groups and institutions to provide
them with the power and resources needed to sustain themselves. For
example, for political regimes to engage in repression they require the
consistent cooperation of police and security forces. If security forces
refuse to carry out orders, then the state will be unable to repress
opposition. Similarly, every regime requires some degree of propaganda
and public-facing narrative. If media refuse to spread regime
narratives, then the regime will be unable to gain the compliance of the
populace. And every citizen helps keep the regime that rules them in
power through paying taxes. All political power rests upon a flow of
these kinds of resources and obedience that must be continually
replenished (\citeproc{ref-sharp-politics-1973}{Sharp 1973}).

Most of the time, regime supporters provide their support in relatively
unchanging ways. Orders are rarely questioned, norms of authority and
deference are followed, and politics proceeds as it always has. These
common patterns of obedience create the illusion that power is stable
and unchanging.

Yet loyalties shift. Out of changing self-interest, or fear of the
future, or moral conviction, the individuals and groups that previously
supported a regime cease to do so. They behave in ways that defy the
wishes of political leaders: soldiers refuse to open fire on
demonstrators, government attorneys refuse to engage in
politically-motivated prosecutions, legislators resign from their
offices, or vote against the leaders or parties that originally brought
them to power.

All these are examples of what we, drawing on an extensive literature,
refer to as \emph{defection}. Defection is any instance in which a
regime supporter significantly and substantively decreases their level
of regime support. It is a critical component of any process of
political change. It is particularly crucial in processes of
\emph{nonviolent} political change, whose logic of change rests not on
eliminating one's opponents but on shifting them to one's side
(\citeproc{ref-chenoweth-why-2011}{Chenoweth and Stephan 2011}).

Consider, as a unifying metaphor, the \emph{spectrum of allies and
opponents} (shown in Figure~\ref{fig-spectrum}), originally developed by
Oppenheimer and Lakey (\citeproc{ref-oppenheimer-manual-1965}{1965}).
The tool places all relevant social and political groups involved a
political struggle into five ordinal positions along a spectrum of
loyalty toward a regime, ranging from active support to active
opposition. This identification process then informs the strategies that
movements employ toward those groups, always seeking to move them from
the right to the left. A \emph{loyalty shift} is any movement from any
point along this spectrum to any other, while \emph{defection}
specifically denotes any left-ward movement that begins in active or
passive support.\footnote{Defection thus overlaps with but is not
  identical to \emph{mobilization}, which is any right-to-left movement
  on the spectrum that ends in the ``Active Opposition'' point.
  Defection overlaps with mobilization when it begins in Passive or
  Active Support and ends in Active Opposition, yet many instances of
  defection do not end in active opposition.}

\begin{figure}

\centering{

\pandocbounded{\includegraphics[keepaspectratio]{defections-framework_files/figure-pdf/fig-spectrum-1.pdf}}

}

\caption{\label{fig-spectrum}Spectrum of Allies and Opponents}

\end{figure}%

\subsection{Scope Conditions}\label{scope-conditions}

Defection is a process that occurs in all political regimes, and in some
sense, within all organizations.\footnote{Hirschman
  (\citeproc{ref-hirschman-exit-1970}{1970}) is one attempt to generate
  a macro-theory of defection relevant across all kinds of
  organizations, with a focus on firms.} For this project, we
specifically examine defection in the context of authoritarian and
democratic backsliding regimes.

This scope condition is both theoretically and practically motivated.
Theoretically, we expect defection in these contexts to systematically
differ from defection in healthy democracies. Repression of opponents is
substantially higher in authoritarian and would-be authoritarian regimes
(\citeproc{ref-davenport-state-2007}{Davenport 2007};
\citeproc{ref-adhikari-examining-2024}{Adhikari, King, and Murdie
2024}), making defection a far riskier action than in democracies. Given
this different risk environment, the personal, social, and political
motivators of defection are likely to be distinct.\footnote{The
  boundaries between these conditions are fuzzy. Scholars debate, for
  instance, what constitutes a period of democratic backsliding
  (\citeproc{ref-little-measuring-2024}{Little and Meng 2024}), and some
  of these periods look very similar to periods of democracy. We define
  democratic backsliding following Jee, Lueders, and Myrick
  (\citeproc{ref-jee-unified-2022}{2022}), 755, as ``any change of a
  political community's formal or informal rules which reduces that
  community's ability to guarantee the freedom of choice, freedom from
  tyranny, or equality in freedom.'' In the 21st century, such periods
  tend to be characterized by ``executive aggrandizement,'' the gradual
  centralization of power in the hands of the executive, often initially
  through legal avenues (\citeproc{ref-bermeo-democratic-2016}{Bermeo
  2016}; \citeproc{ref-riedl-pathways-2024}{Riedl et al. 2024}).
  Operationally, our starting point for identifying periods of
  democratic backsliding is the definition of autocratization episodes
  from Lührmann and Lindberg (\citeproc{ref-luhrmann-third-2019}{2019})}

In practical terms, this project's core motivation is to better
understand how to protect, repair, and advance democracy. Thus the
factors that motivate lower-risk defection in healthy democracies are of
less interest. We do not rule out the possibility that examining
defection in democratic settings may yield useful insights, nor do we
think that our findings would be irrelevant for better understanding
defection in healthy democracy.

We also focus on defection in the contemporary world, with particular
attention to the post-Cold War period. This is because, as the
literature has shown, the logic of both authoritarian rule and
democratic backsliding is significantly different today than in the
past. Authoritarian regimes today are much more likely to have
pseudo-democratic institutions, and to rely less on the most direct and
brutal forms of physical repression
(\citeproc{ref-guriev-informational-2019}{Guriev and Treisman 2019};
\citeproc{ref-levitsky-competitive-2010}{Levitsky and Way 2010}).
Historically, democratic backsliding tended to occur in sharp, dramatic
breaks (\citeproc{ref-linz-breakdown-1978}{Linz and Stepan 1978}).
Today, gradual processes of decline centered in aggrandizement of the
executive and conducted under a veneer of legality are more common
(\citeproc{ref-bermeo-democratic-2016}{Bermeo 2016};
\citeproc{ref-riedl-democratic-2025}{Riedl et al. 2025}).

\section{What Do We Already Know?}\label{what-do-we-already-know}

Different forms of defection have largely been studied in isolation,
without a unifying framework. Many scholars have examined defection by
the military or other security forces, particularly in the context of
ongoing civil resistance campaigns
(\citeproc{ref-nepstad-mutiny-2013}{Nepstad 2013};
\citeproc{ref-morency-laflamme-question-2018}{Morency-Laflamme 2018};
\citeproc{ref-grewal-military-2019}{Grewal 2019};
\citeproc{ref-neu-strategies-2022}{Neu 2022};
\citeproc{ref-dahl-disaggregating-2026}{Dahl, Celestino, and Gates
2026}; \citeproc{ref-pion-berlin-staying-2014}{Pion-Berlin, Esparza, and
Grisham 2014}). Others have examined defection by individual elites or
specific elite groups. For instance, there is a significant literature
investigating why and how legislators in authoritarian single-party
regimes leave the ruling party (\citeproc{ref-delrio-strategic-2022}{del
Río 2022}; \citeproc{ref-reuter-economic-2011}{Reuter and Gandhi 2011};
\citeproc{ref-reuter-elite-2019}{Reuter and Szakonyi 2019}), and a
smaller body of work examining why individual authoritarian regime
supporters withdraw their support (\citeproc{ref-hale-who-2017}{Hale and
Colton 2017}; \citeproc{ref-tertytchnaya-protests-2020}{Tertytchnaya
2020}).

While this fragmented literature comes to a variety of conclusions, a
few consensus findings stand out:

First, defection is \emph{consequential}. The civil resistance
literature is a central touchstone here. Defections by security forces
in particular significantly increase the likelihood that a maximalist
civil resistance campaign will succeed
(\citeproc{ref-chenoweth-why-2011}{Chenoweth and Stephan 2011}); civil
resistance strategies aimed at inducing defections among regime
supporters have been shown to outperform strategies focused solely on
mass mobilization (\citeproc{ref-chenoweth-dynamic-2022}{Chenoweth,
Hocking, and Marks 2022}). Beyond civil resistance, divisions in elite
coalitions are one of the most consequential precursors of
democratization (\citeproc{ref-odonnell-transitions-1986}{O'Donnell and
Schmitter 1986}; \citeproc{ref-djuve-patterns-2019}{Djuve, Knutsen, and
Wig 2019})\footnote{Though the effects of such divisions may be
  contingent on regime characteristics
  (\citeproc{ref-delrio-democratic-2025}{del Río and Higashijima 2025})}

Second, defection is \emph{difficult}. Loyalty to one's group, even at
the expense of personal self-interest, is a basic characteristic of
human psychology (\citeproc{ref-billig-social-1973}{Billig and Tajfel
1973}). Thus, defection faces inherent psychological barriers. These
barriers are reinforced by deliberate regime strategies. Authoritarian
leaders recognize the threat posed by defection and invest in strategies
to prevent it (\citeproc{ref-svolik-politics-2012}{Svolik 2012};
\citeproc{ref-markowitz-kompromat-2017}{Markowitz 2017}), from designing
institutional structures to co-opt potential defectors or counterbalance
them against one another (\citeproc{ref-gandhi-political-2008}{Gandhi
2008}; \citeproc{ref-debruin-how-2020}{De Bruin 2020}) to censorship of
nascent collective action (\citeproc{ref-king-how-2013}{King, Pan, and
Roberts 2013}), to unleashing state repression
(\citeproc{ref-fruge-repressive-2019}{Frugé 2019}). Together, these
psychological and practical barriers make defection rare. In del Río
(\citeproc{ref-delrio-strategic-2022}{2022})'s study of defection among
legislative elites in authoritarian regimes, only 3\% of the individuals
studied ever defected, even in the final days of a regime. Similarly, in
a study of democratic backsliding spells, Pinckney and Trilling
(\citeproc{ref-pinckney-breaking-2025}{2025}) find that support groups
for democratic backsliding regimes defected \emph{en masse} in only a
small proportion of backsliding spells.

Third, defection is \emph{complex but predictable}. In a foundational
contribution, Kuran (\citeproc{ref-kuran-now-1991}{1991}) argued that
defection from authoritarian regimes is inherently unpredictable due to
near-universal preference falsification. This prevents an outside
observer from knowing beforehand when individuals would be willing to
speak out against the regime. When change did occur, Kuran argued it was
likely to come about as a sudden and shocking ``cascade.'' While Kuran's
argument remains an important touchstone, subsequent research has
highlighted that while defection is complex, often nonlinear, and
challenging to study in authoritarian contexts, it does have reliable
predictors. Economic self-interest is a consistent theme among these
predictors. Both poor economic performance at the national level, and
declining personal economic fortunes in the authoritarian regime are
strong predictors of defection
(\citeproc{ref-reuter-economic-2011}{Reuter and Gandhi 2011};
\citeproc{ref-grewal-military-2019}{Grewal 2019};
\citeproc{ref-pion-berlin-staying-2014}{Pion-Berlin, Esparza, and
Grisham 2014}). Widespread defection taking place in neighboring
countries is another
(\citeproc{ref-beissinger-structure-2007}{Beissinger 2007};
\citeproc{ref-gleditsch-diffusion-2017}{Gleditsch and Rivera 2017};
\citeproc{ref-bunce-defeating-2011}{Bunce and Wolchik 2011}). Scholars
have found that institutional structures that make defection more costly
tend to make it less likely
(\citeproc{ref-langston-breaking-2002}{Langston 2002}).

\section{A Few Guiding Assumptions}\label{a-few-guiding-assumptions}

Given the conceptual breadth of defections, even within the empirical
scope limitations described above there are numerous ways to study the
phenomenon. In this section we lay out a few of the core theoretical
assumptions underlying our approach.

\subsection{Defection Can Be Individual or
Collective}\label{defection-can-be-individual-or-collective}

Defection occurs at both the individual (micro) level and group (meso)
level. Individuals may make personal decisions to defect that do not
substantively impact the social and political groups to which they
belong. Groups can also defect collectively in ways that cannot be
reduced to a simple aggregation of individual decisions. Individuals are
embedded within social groups, and groups are embedded in large social
and political systems. While the aggregation of individual-level
characteristics for all members of a group can tell us something about
the group's likely behavior, it is unlikely to capture the whole story.
Likewise, knowing the groups to which an individual belongs can help
predict propensity to defect, but is unlikely to tell us everything.

This means there is interesting and important research to be done on
defection both at the individual and the group level, and research that
ignores one of these levels is likely to yield an incomplete picture.

\subsection{Defection is Social}\label{defection-is-social}

Defection cannot be studied in isolation. While unit-level factors
influence a person or group's propensity to defect, defection unfolds in
an inescapably social context. Individuals and groups decide whether or
not to defect not just based on what they want but on what they think
other people and groups are likely to do
(\citeproc{ref-koehler-disaffection-2016}{Koehler, Ohl, and Albrecht
2016}). Understanding defection therefore requires attention not only to
characteristics of individuals or groups but also to their embeddedness
within broader social networks and the flows of information through
those networks.

\subsection{Defection Has Multi-Faceted, Obscure
Motivations}\label{defection-has-multi-faceted-obscure-motivations}

Individuals or groups defect for many reasons. These may be stated or
unstated, and may be transparent or opaque even to the defectors
themselves. In some cases, defection may be a matter of rational,
material utility maximization; in others it may be a matter of moral
conviction or suasion (\citeproc{ref-pearlman-moral-2018}{Pearlman
2018}). Or it may be motivated by changing norms of appropriate behavior
in one's social group. These motivations are not sharply segmented from
one another, and frequently interact. Material interests are easier to
advance if one can morally justify them. Conversely, commitment to one's
principles is easier to sustain if it also happens be materially
beneficial.

The interrelated and often ambiguous nature of defection motivations
implies that we should treat self-reported explanations as one data
source among many, not as the definitive answer to \emph{why} defection
occurred. The stories people tell about why they made consequential
decisions are not simply direct replications of their actual
motivations, but are narratives intended to shape a self or collective
image (\citeproc{ref-nisbett-telling-1977}{Nisbett and Wilson 1977}).

This leads directly to the following guiding assumption:

\subsection{Defection is Behavioral}\label{defection-is-behavioral}

Defection is defined not by what people \emph{think} but by what they
\emph{do}. For defection to occur, there must be an observable
behavioral change vis a vis an authoritarian regime. Individuals or
groups must stop providing forms of support they previously provided.
While changes in beliefs or attitudes may be important precursors to
defection, if they do not result in meaningful, observable behavior
change they do not constitute defection.

This does not mean we are uninterested in public expressions of loyalty
or dissent toward an authoritarian regime. Speech is an important aspect
of behavior, particularly in authoritarian contexts where even the
slightest indication of dissent is politically meaningful
(\citeproc{ref-hale-who-2017}{Hale and Colton 2017}). Statements that
reflect a dissident attitude toward a regime are important behavioral
changes and can lead to transformational consequences
(\citeproc{ref-kuran-now-1991}{Kuran 1991}). In Communist Romania, the
simple act of a crowd publicly booing a speech by dictator Nicolae
Ceausescu led directly to the downfall of the regime.

At the same time, what constitutes defection is not the same across
contexts. Whether a particular behavior is a defection will depend on
what, for that individual or group, constitutes loyalty. Some behaviors
may be clearly defection across almost all contexts, but in most cases
identifying defection and assessing its significance will require
significant contextual knowledge.\footnote{See Pinckney
  (\citeproc{ref-pinckney-nonviolent-2021}{2021}) for a similar
  discussion of what constitutes nonviolent resistance. A public march,
  for instance, by an excluded group in a dictatorship where such
  marches are criminalized, is a bold act of nonviolent resistance. A
  tactically identical march by a privileged group in a democracy may
  simply be politics as usual.}

Vaclav Havel (\citeproc{ref-havel-power-1985}{1985})'s seminal essay
``The Power of the Powerless'' illustrates this point. Havel describes
the momentousness of a shopkeeper in Communist Czechoslovakia choosing
to remove a ``Workers of the World, Unite'' sign from his shop window.
Havel describes how the universality of such displays creates the
metaphysical structure that gives the post-totalitarian Communist regime
its power, rendering even minor acts of defiance politically
consequential. However, to recognize the importance of such an action,
we must understand the routine behaviors expected of an average citizen
of this regime and how people interpret deviations from those behaviors.

This discussion leads into our final guiding theoretical assumption:

\subsection{Defection Takes Many
Forms}\label{defection-takes-many-forms}

The ways in which defection happens are as diverse as the ways that
regime supporters display loyalty. Defection will look different across
individuals and groups. It can involve incremental steps or dramatic
steps, can be publicized to the world or confided within private
networks of trusted friends and confidants.

As a first step towards theoretical clarity, we categorize forms of
defection as falling along two dimensions: whether they \emph{break} or
\emph{bind} with the regime and whether the involve \emph{speaking,
acting,} or \emph{standing in the way.}

The first dimension captures whether the defection is intended to change
the regime from within or from outside.\footnote{This dimension has
  close parallels to the idea of \emph{insider} or \emph{outsider}
  strategies for achieving social change, as well as Hirschman's classic
  distinction between ``voice'' and ``exit''
  (\citeproc{ref-hirschman-exit-1970}{Hirschman 1970})} Defection that
\emph{breaks} typically involves severing ties with the regime, often
through highly visible actions or explicit denunciations. In social
psychological terms, it involves a definitive disconnection from a
social identity once previously held. Defection that \emph{binds}, by
contrast, seeks to maintain one's prior social identity as a part of a
regime while applying internal pressure to motivate change in the
regime's behavior. Such defection typically involves behaviors that are
quiet, behind the scenes, and perhaps only visible in their effects.

The second dimension refers to the defector's specific behaviors.
Defection through \emph{speaking} involves verbal expression of
opposition to the regime. Defection through \emph{acting} entails taking
positive steps against the regime. Defection through \emph{standing in
the way} consists of refusing to do what is normally
expected.\footnote{This dimension roughly parallels Gene Sharp
  (\citeproc{ref-sharp-politics-1973}{1973})'s division of the methods
  of nonviolent action into \emph{protest and persuasion} (speaking),
  \emph{non-cooperation} (standing in the way), and \emph{nonviolent
  intervention} (acting), with some conceptual overlap and some
  distinctions. While many instances of defection are likely also
  instances of nonviolent action/nonviolent resistance, some are not.}
Table~\ref{tbl-matrix} describes some indicative examples of different
forms of defections along these two dimensions.

\begin{table}

\caption{\label{tbl-matrix}Types of Defection}

\centering{

\begin{tabular}{>{}lll}
\toprule
\textbf{} & \textbf{Breaking} & \textbf{Binding}\\
\midrule
\textbf{Speaking} & Public Condemnation & Private Disapproval\\
\textbf{Acting} & Joining a Demonstration & Creating Internal Reforms\\
\textbf{Standing in the Way} & Disobeying Orders & Foot-Dragging\\
\bottomrule
\end{tabular}

}

\end{table}%

As defection unfolds over time will, it will likely begin with binding
and then breaking, and may involve a combination speaking, acting, and
standing in the way. Specific acts of defection by the same group or
individual can be distinguished based on whether they vary across these
dimensions.

\section{A Few Propositions}\label{a-few-propositions}

Having laid out our guiding assumptions about defection, we now describe
a set of initial hypotheses that are warrant further investigation. We
are particularly interested in the ways in which action by opposition
groups, civil resistance campaigns, and others struggling against
authoritarian or backsliding regimes can facilitate or spark defection.
Accordingly, the hypotheses that follow are generally framed within
these contexts.

\subsection{Defection Requires a Way
Out}\label{defection-requires-a-way-out}

Regime supporters will be highly unlikely to defect if they do not
perceive an acceptable post-defection future. For example, Sinanoglu
(\citeproc{ref-sinanoglu-autocrats-2025}{2025}) points out that
corruption can powerfully deter defection among business elites. Not
only do businesses that have corrupt relationships with authoritarian
regimes fear losing the material benefits of corruption, they also know
that their corruption engenders animosity toward them among the
opposition. Thus, anticipating adverse consequences in a post-regime
world, business elites may remain loyal to the regime even when it is
not in their material interest.

Defection is therefore most likely to occur when potential defectors can
identify viable avenues to achieve their core individual or
organizational goals post-defection. This is partially a matter of
institutional structure that may be outside of activists' immediate
control - if the existing social, political, or economic institutions
provide no meaningful prospects for post-defection life, then defection
will be particularly rare
(\citeproc{ref-langston-breaking-2002}{Langston 2002}). Eliminating such
alternative pathways is a key strategy of authoritarian control.

However, skillful resistance campaigns can counter this strategy by
promoting ``ways out'' for potential defectors. They may do so by
creating alternative institutions, negotiating exit plans, and by
promoting an inclusive vision of a post-regime future in which potential
defectors can see themselves.

\subsection{It's Not What You Know, It's Who You
Know}\label{its-not-what-you-know-its-who-you-know}

Defection is likely to be driven by social connections. Individuals and
groups behave not only according to a material ``logic of consequences''
but also according to a ``logic of appropriateness'' for what people and
groups ``like them'' typically do
(\citeproc{ref-march-institutional-1998}{March and Olsen 1998}). All
else equal, any factor that increases a regime supporter's positive
connections to individuals and groups that have publicly-known lower
levels of loyalty to the regime will tend to reduce that supporter's
regime loyalty and increase the probability of defection. This mechanism
also connects to the social psychological dynamics of in-group loyalty
that make defection difficult. When the boundaries of the relevant
``group'' are blurred, such that regime supporters come to see their
social group as including both regime supporters and opponents, then
in-group loyalty will no longer necessarily imply \emph{regime} loyalty.

For this mechanism to operate, lower levels of regime loyalty among a
supporter's social group must be \emph{known}. Thus, factors that cause
regime supporters to update their priors about the level of dissent
against the regime, particularly among unlikely dissenters, is likely to
increase the probability of defection.

For resistance campaigns, this implies that defection should be more
likely under two conditions: (1) when the campaign increases their
social connection to regime supporters and (2) when they effectively
signal that dissent is both more widespread and more durable than regime
supporters previously anticipated.

\subsection{\ldots but Material Consequences
Matter}\label{but-material-consequences-matter}

While social connections are important, they do not tell the whole
story. Defection in authoritarian contexts is also a high-risk,
high-cost endeavor, and regime supporters are likely to be sensitive to
the costs of their potential actions. Thus, factors that increase the
personal or group cost of remaining loyal to the regime relative to the
cost of defection are likely to increase the scale and scope of
defection. To avoid overly stretching the concept of cost we focus on
material costs and benefits.

Resistance campaigns can heighten the costs of loyalty through tactics
of non-cooperation or nonviolent intervention that target the material
bases of support for regime loyalists.

There is a tension between the social logic for promoting defection
highlighted in the previous hypothesis and the material logic of
promoting defection emphasized here. To illustrate, consider a business
that is providing support to an authoritarian leader. An
anti-authoritarian campaign could attempt to spur defection by
encouraging its members to patronize the business, thereby building
strong social connections between the business and the campaign's
members and developing a social logic of appropriateness that might
motivate the business to defect. Alternatively, the campaign could
organize a boycott, cutting off social connections between itself and
the business, but heightening the material costs of loyalty until
certain concessions are made. Which strategy should the campaign pursue?

We are skeptical that either the social or the material logic of
sparking defection predominates across all contexts. Instead, effective
campaigns are likely to spark defection through a carefully sequenced
combination of tactics that leverage both logics. Tactics that attempt
to impose material costs are more likely to succeed when they are built
on existing social connections and clear communication. In other words,
material pressure may be most effective when the imposition of costs is
seen as a temporary break in a relationship that can be restored based
on good behavior.

Future research should explore the specific parameters under which these
logics can be most effectively employed, how they are best combined, and
how they interact with other consequential factors such as the relative
power, pre-existing relationships, and institutional position of regime
supporters.

\subsection{Little Steps are Easier than Big
Steps}\label{little-steps-are-easier-than-big-steps}

We expect that defection is generally easier among those with the lowest
levels of support for a regime, and that small-scale defections are far
more common than large-scale defections. In the framework of the
spectrum of allies and opponents, we expect that the most common type of
defection is a shift from passive support to neutrality. The second most
common will be moves from active support to passive support. Jumps of
more than one level - most extremely from active support to active
opposition - will be much less common.\footnote{For evidence supporting
  this general framework see Neundorf et al.
  (\citeproc{ref-neundorf-loyal-2025}{2025}) or Tertytchnaya
  (\citeproc{ref-tertytchnaya-protests-2020}{2020}).}

For active supporters to become active opponents of a regime multiple
steps will thus typically be required. Active supporters will typically
move from active to passive support, then to neutrality, followed by
passive opposition, before (if ever) ultimately reaching active
opposition.

\subsection{\ldots but authoritarianism can make defection all or
nothing}\label{but-authoritarianism-can-make-defection-all-or-nothing}

Authoritarian and would-be authoritarian regimes have strong incentives
to prevent the gradual declines in loyalty described in the previous
hypothesis. They are highly likely to punish even modest indications of
declining loyalty with severe repression. As a result, supporters whose
private loyalties are shifting will be more hesitant to express those
shifts. This means that over time authoritarian and would-be
authoritarian regimes accumulate large numbers of public supporters
whose private preferences strongly oppose them
(\citeproc{ref-kuran-now-1991}{Kuran 1991}). When these individuals
defect, they are likely to do so in dramatic, all-or-nothing ways.

Thus, we hypothesize that the scale of defection is directly related to
the degree of authoritarianism in the current political regime. In more
open political environments, we will tend to observe more small
defections, such as moves from passive support to neutrality. As the
system becomes more authoritarian, defection will become less frequent,
but larger and more abrupt when it occurs. We will see more instances of
active regime supporters moving to active regime opponents.

\section{Conclusion}\label{conclusion}

All political regimes are built on loyalty. Understanding changes in
that loyalty is key to understanding regime stability and
transformation. For democratization to occur, or for democratic
backsliding to be checked, the individuals and groups that support an
authoritarian leader must cease to provide that support. They must
defect.

This project, grounded in the conceptual framework laid out in this
document, seeks to unify the disparate ways scholars have studied
previously defection and advance new insights into when, how, and why
defection occurs. By doing so, this project aims to provide actionable
recommendations to activists and policymakers seeking to advance,
defend, and repair democracy around the world.

\section*{References}\label{references}
\addcontentsline{toc}{section}{References}

\phantomsection\label{refs}
\begin{CSLReferences}{1}{0}
\bibitem[\citeproctext]{ref-adhikari-examining-2024}
Adhikari, Bimal, Jeffrey King, and Amanda Murdie. 2024. {``Examining the
Effects of Democratic Backsliding on Human Rights Conditions.''}
\emph{Journal of Human Rights} 23 (3): 267--82.
\url{https://doi.org/10.1080/14754835.2023.2295878}.

\bibitem[\citeproctext]{ref-beissinger-structure-2007}
Beissinger, Mark R. 2007. {``Structure and Example in Modular Political
Phenomena: {The} Diffusion of Bulldozer/Rose/Orange/Tulip
Revolutions.''} \emph{Perspectives on Politics} 5 (2): 259--76.

\bibitem[\citeproctext]{ref-bermeo-democratic-2016}
Bermeo, Nancy. 2016. {``On {Democratic Backsliding}.''} \emph{Journal of
Democracy} 27 (1): 5--19. \url{https://doi.org/10.1353/jod.2016.0012}.

\bibitem[\citeproctext]{ref-billig-social-1973}
Billig, Michael, and Henri Tajfel. 1973. {``Social Categorization and
Similarity in Intergroup Behaviour.''} \emph{European Journal of Social
Psychology} 3 (1): 27--52.
\url{https://doi.org/10.1002/ejsp.2420030103}.

\bibitem[\citeproctext]{ref-bunce-defeating-2011}
Bunce, Valerie J., and Sharon L. Wolchik. 2011. \emph{Defeating
{Authoritarian Leaders} in {Postcommunist Countries}}. New York, NY:
Cambridge University Press.

\bibitem[\citeproctext]{ref-chenoweth-dynamic-2022}
Chenoweth, Erica, Andrew Hocking, and Zoe Marks. 2022. {``A Dynamic
Model of Nonviolent Resistance Strategy.''} \emph{PLOS ONE} 17 (7):
e0269976. \url{https://doi.org/10.1371/journal.pone.0269976}.

\bibitem[\citeproctext]{ref-chenoweth-why-2011}
Chenoweth, Erica, and Maria J. Stephan. 2011. \emph{Why Civil Resistance
Works: {The} Strategic Logic of Nonviolent Conflict}. New York, NY:
Columbia University Press.

\bibitem[\citeproctext]{ref-dahl-disaggregating-2026}
Dahl, Marianne, Mauricio Rivera Celestino, and Scott Gates. 2026.
{``Disaggregating {Defection}: {Dissent Campaign Strategies} and
{Security Force Disloyalty}.''} \emph{Journal of Conflict Resolution} 70
(1): 170--95. \url{https://doi.org/10.1177/00220027251348389}.

\bibitem[\citeproctext]{ref-davenport-state-2007}
Davenport, Christian. 2007. \emph{State Repression and the Domestic
Democratic Peace}. Cambridge University Press.

\bibitem[\citeproctext]{ref-debruin-how-2020}
De Bruin, Erica. 2020. \emph{How to {Prevent Coups} d'{Etat}:
{Counterbalancing} and {Regime Survival}}. Ithaca, NY: Cornell
University Press.

\bibitem[\citeproctext]{ref-delrio-strategic-2022}
del Río, Adrián. 2022. {``Strategic {Uncertainty} and {Elite Defections}
in {Electoral Autocracies}: {A Cross-National Analysis}.''}
\emph{Comparative Political Studies} 55 (13): 2250--82.
\url{https://doi.org/10.1177/00104140221074273}.

\bibitem[\citeproctext]{ref-delrio-democratic-2025}
del Río, Adrián, and Masaaki Higashijima. 2025. {``Democratic {Reforms}
in {Dictatorships}: {Elite Divisions}, {Party Origins}, and the
{Prospects} of {Political Liberalization}.''} \emph{Comparative
Political Studies} 58 (12): 2682--2717.
\url{https://doi.org/10.1177/00104140241302772}.

\bibitem[\citeproctext]{ref-djuve-patterns-2019}
Djuve, Vilde Lunnan, Carl Henrik Knutsen, and Tore Wig. 2019.
{``Patterns of {Regime Breakdown Since} the {French Revolution}.''}
\emph{Comparative Political Studies} 53 (6): 923--58.
\url{https://doi.org/10.1177/0010414019879953}.

\bibitem[\citeproctext]{ref-fruge-repressive-2019}
Frugé, Kimberly R. 2019. {``Repressive Agent Defections: {How} Power,
Costs, and Uncertainty Influence Military Behavior and State
Repression.''} \emph{Conflict Management and Peace Science} 36 (6):
591--607. \url{https://doi.org/10.1177/0738894219881433}.

\bibitem[\citeproctext]{ref-gandhi-political-2008}
Gandhi, Jennifer. 2008. \emph{Political {Institutions} Under
{Dictatorship}}. Cambridge: Cambridge University Press.
\url{https://doi.org/10.1017/CBO9780511510090}.

\bibitem[\citeproctext]{ref-gleditsch-diffusion-2017}
Gleditsch, Kristian S., and Mauricio Rivera. 2017. {``The Diffusion of
Nonviolent Campaigns.''} \emph{Journal of Conflict Resolution} 61 (5):
1120--45.

\bibitem[\citeproctext]{ref-grewal-military-2019}
Grewal, Sharan. 2019. {``Military Defection During Localized Protests:
{The} Case of {Tataouine}.''} \emph{International Studies Quarterly} 63
(2): 259--69.

\bibitem[\citeproctext]{ref-guriev-informational-2019}
Guriev, Sergei, and Daniel Treisman. 2019. {``Informational
{Autocrats}.''} \emph{Journal of Economic Perspectives} 33 (4):
100--127. \url{https://doi.org/10.1257/jep.33.4.100}.

\bibitem[\citeproctext]{ref-hale-who-2017}
Hale, Henry E., and Timothy J. Colton. 2017. {``Who {Defects}?
{Unpacking} a {Defection Cascade} from {Russia}'s {Dominant Party}
2008--12.''} \emph{American Political Science Review} 111 (2): 322--37.
\url{https://doi.org/10.1017/S0003055416000642}.

\bibitem[\citeproctext]{ref-havel-power-1985}
Havel, Vaclav. 1985. {``The {Power} of the {Powerless}.''} In \emph{The
{Power} of the {Powerless}: {Citizens Against} the {State} in
{Central-Eastern Europe}}, edited by John Keane, 23--96. Armonk, NY:
M.E. Sharpe, Inc.

\bibitem[\citeproctext]{ref-hirschman-exit-1970}
Hirschman, Albert O. 1970. \emph{Exit, {Voice}, and {Loyalty}:
{Responses} to {Decline} in {Firms}, {Organizations}, and {States}}.
Cambridge, MA: Harvard University Press.

\bibitem[\citeproctext]{ref-jee-unified-2022}
Jee, Haemin, Hans Lueders, and Rachel Myrick. 2022. {``Towards a Unified
Approach to Research on Democratic Backsliding.''}
\emph{Democratization} 29 (4): 754--67.
\url{https://doi.org/10.1080/13510347.2021.2010709}.

\bibitem[\citeproctext]{ref-king-how-2013}
King, Gary, Jennifer Pan, and Margaret E. Roberts. 2013. {``How
Censorship in {China} Allows Government Criticism but Silences
Collective Expression.''} \emph{American Political Science Review} 107
(2): 326--43.

\bibitem[\citeproctext]{ref-koehler-disaffection-2016}
Koehler, Kevin, Dorothy Ohl, and Holger Albrecht. 2016. {``From
{Disaffection} to {Desertion}: {How Networks Facilitate Military
Insubordination} in {Civil Conflict}.''} \emph{Comparative Politics} 48
(4): 439--57. \url{https://doi.org/10.5129/001041516819197601}.

\bibitem[\citeproctext]{ref-kuran-now-1991}
Kuran, Timur. 1991. {``Now Out of Never: {The} Element of Surprise in
the {East European} Revolution of 1989.''} \emph{World Politics} 44 (1):
7--48.

\bibitem[\citeproctext]{ref-langston-breaking-2002}
Langston, Joy. 2002. {``Breaking {Out} Is {Hard} to {Do}: {Exit},
{Voice}, and {Loyalty} in {Mexico}'s {One-Party Hegemonic Regime}.''}
\emph{Latin American Politics and Society} 44 (3): 61--88.
\url{https://doi.org/10.1111/j.1548-2456.2002.tb00214.x}.

\bibitem[\citeproctext]{ref-levitsky-competitive-2010}
Levitsky, Steven, and Lucan A. Way. 2010. \emph{Competitive
Authoritarianism: {Hybrid} Regimes After the {Cold War}}. New York, NY:
Cambridge University Press.

\bibitem[\citeproctext]{ref-linz-breakdown-1978}
Linz, Juan J., and Alfred Stepan, eds. 1978. \emph{The {Breakdown} of
{Democratic Regimes}}. Johns Hopkins University Press.

\bibitem[\citeproctext]{ref-little-measuring-2024}
Little, Andrew T., and Anne Meng. 2024. {``Measuring Democratic
Backsliding.''} \emph{PS: Political Science \& Politics}, 1--13.

\bibitem[\citeproctext]{ref-luhrmann-third-2019}
Lührmann, Anna, and Staffan I. Lindberg. 2019. {``A Third Wave of
Autocratization Is Here: What Is New about It?''} \emph{Democratization}
26 (7): 1095--1113.

\bibitem[\citeproctext]{ref-march-institutional-1998}
March, James G., and Johan P. Olsen. 1998. {``The Institutional Dynamics
of International Political Orders.''} \emph{International Organization}
52 (4): 943--69.

\bibitem[\citeproctext]{ref-markowitz-kompromat-2017}
Markowitz, Lawrence P. 2017. {``Beyond {Kompromat}: {Coercion},
{Corruption}, and {Deterred Defection} in {Uzbekistan}.''}
\emph{Comparative Politics} 50 (1): 103--21.
\url{https://doi.org/10.5129/001041517821864390}.

\bibitem[\citeproctext]{ref-morency-laflamme-question-2018}
Morency-Laflamme, Julien. 2018. {``A Question of Trust: Military
Defection During Regime Crises in {Benin} and {Togo}.''}
\emph{Democratization} 25 (3): 464--80.
\url{https://doi.org/10.1080/13510347.2017.1375474}.

\bibitem[\citeproctext]{ref-nepstad-mutiny-2013}
Nepstad, Sharon Erickson. 2013. {``Mutiny and Nonviolence in the {Arab
Spring}: {Exploring} Military Defections and Loyalty in {Egypt},
{Bahrain}, and {Syria}.''} \emph{Journal of Peace Research} 50 (3):
337--49.

\bibitem[\citeproctext]{ref-neu-strategies-2022}
Neu, Kara Kingma. 2022. {``Strategies of {Dictatorship} and {Military
Disloyalty} During {Anti-Authoritarian Protests}: {Explaining
Defections} and {Coups}.''} \emph{Journal of Global Security Studies} 7
(1): ogab033. \url{https://doi.org/10.1093/jogss/ogab033}.

\bibitem[\citeproctext]{ref-neundorf-loyal-2025}
Neundorf, Anja, Aykut Ozturk, Ksenia Northmore-Ball, Katerina
Tertytchnaya, and Johannes Gerschewski. 2025. {``A {Loyal Base}:
{Support} for {Authoritarian Regimes} in {Times} of {Crisis}.''}
\emph{Comparative Political Studies} 58 (9): 1854--89.
\url{https://doi.org/10.1177/00104140241283006}.

\bibitem[\citeproctext]{ref-nisbett-telling-1977}
Nisbett, Richard E., and Timothy D. Wilson. 1977. {``Telling More Than
We Can Know: {Verbal} Reports on Mental Processes.''}
\emph{Psychological Review} 84 (3): 231.

\bibitem[\citeproctext]{ref-odonnell-transitions-1986}
O'Donnell, Guillermo, and Philippe C. Schmitter. 1986. \emph{Transitions
from Authoritarian Rule: {Tentative} Conclusions about Uncertain
Democracies}. Baltimore, MD: JHU Press.

\bibitem[\citeproctext]{ref-oppenheimer-manual-1965}
Oppenheimer, Martin, and George Lakey. 1965. \emph{A {Manual} for
{Direct Action}}. Quadrangle Books.

\bibitem[\citeproctext]{ref-pearlman-moral-2018}
Pearlman, Wendy. 2018. {``Moral {Identity} and {Protest Cascades} in
{Syria}.''} \emph{British Journal of Political Science} 48 (4):
877--901. \url{https://doi.org/10.1017/S0007123416000235}.

\bibitem[\citeproctext]{ref-pinckney-nonviolent-2021}
Pinckney, Jonathan. 2021. {``Nonviolent {Resistance}, {Social Justice},
and {Positive Peace}.''} In \emph{The {Palgrave Handbook} on {Positive
Peace}}, edited by Katerina Standish, Heather Devere, Adan Suazo, and
Rachel Rafferty. Palgrave Macmillan.

\bibitem[\citeproctext]{ref-pinckney-breaking-2025}
Pinckney, Jonathan, and Claire Trilling. 2025. {``Breaking {Down
Pillars} of {Support} for {Democratic Backsliding}.''}
\emph{Mobilization: An International Quarterly} 30 (2): 171--88.

\bibitem[\citeproctext]{ref-pion-berlin-staying-2014}
Pion-Berlin, David, Diego Esparza, and Kevin Grisham. 2014. {``Staying
{Quartered}: {Civilian Uprisings} and {Military Disobedience} in the
{Twenty-First Century}.''} \emph{Comparative Political Studies} 47 (2):
230--59. \url{https://doi.org/10.1177/0010414012450566}.

\bibitem[\citeproctext]{ref-reuter-economic-2011}
Reuter, Ora John, and Jennifer Gandhi. 2011. {``Economic {Performance}
and {Elite Defection} from {Hegemonic Parties}.''} \emph{British Journal
of Political Science} 41 (1): 83--110.
\url{https://doi.org/10.1017/S0007123410000293}.

\bibitem[\citeproctext]{ref-reuter-elite-2019}
Reuter, Ora John, and David Szakonyi. 2019. {``Elite Defection Under
Autocracy: {Evidence} from {Russia}.''} \emph{American Political Science
Review} 113 (2): 552--68.

\bibitem[\citeproctext]{ref-riedl-democratic-2025}
Riedl, Rachel Beatty, Paul Friesen, Jennifer McCoy, and Kenneth Roberts.
2025. {``Democratic {Backsliding}, {Resilience}, and {Resistance}.''}
\emph{World Politics} 77 (1): 151--78.

\bibitem[\citeproctext]{ref-riedl-pathways-2024}
Riedl, Rachel Beatty, Jennifer McCoy, Kenneth Roberts, and Murat Somer.
2024. {``Pathways of {Democratic Backsliding}, {Resistance}, and
({Partial}) {Recoveries}.''} \emph{The ANNALS of the American Academy of
Political and Social Science} 712 (1): 8--31.
\url{https://doi.org/10.1177/00027162251319909}.

\bibitem[\citeproctext]{ref-sharp-politics-1973}
Sharp, Gene. 1973. \emph{The {Politics} of {Nonviolent Action}}. Boston,
MA: Porter Sargent.

\bibitem[\citeproctext]{ref-sinanoglu-autocrats-2025}
Sinanoglu, Semuhi. 2025. {``Autocrats and {Their Business Allies}: {The
Informal Politics} of {Defection} and {Co-optation}.''} \emph{Government
and Opposition} OnlineFirst.

\bibitem[\citeproctext]{ref-svolik-politics-2012}
Svolik, Milan W. 2012. \emph{The Politics of Authoritarian Rule}. New
York, NY: Cambridge University Press.

\bibitem[\citeproctext]{ref-tertytchnaya-protests-2020}
Tertytchnaya, Katerina. 2020. {``Protests and {Voter Defections} in
{Electoral Autocracies}: {Evidence From Russia}.''} \emph{Comparative
Political Studies} 53 (12): 1926--56.
\url{https://doi.org/10.1177/0010414019843556}.

\end{CSLReferences}




\end{document}
